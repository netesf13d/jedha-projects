
\documentclass[11pt,a4paper,computermodern]{article}


\usepackage[
%	includeheadfoot,
	width=172mm,
	top=18mm,
	bottom=18mm,
	bindingoffset=4mm
	]{geometry}



%%% Typeface packages
\usepackage[utf8]{inputenc}
\usepackage[T1]{fontenc}
\usepackage{fontawesome}

\usepackage{multirow}
\usepackage{tabularx}
\usepackage{booktabs}
\usepackage[flushleft]{threeparttable}
\usepackage{enumitem}

%%% Graphics packages
\usepackage{graphicx}
\usepackage{subcaption}


%%

\usepackage{hyperref}
\hypersetup{
	colorlinks=false,
	hidelinks=true,
}


%%
\title{Spotify Data Governance Framework Implementation}
\date{}


\begin{document}

\maketitle

\vspace{-10mm}

This document outlines how Spotify should operationalize its Data Governance Framework, including the organizational model, tools, pilot plan, and execution steps.

\section*{Organizational Model}

Given the size of the company and the amount of data to manage, the recommended approach is that of the center of excellence (CoE) model. In this model, a central data governance team, lead by the CDO, defines the standards and practices for data management. It also provides training and support to the data teams embedded in each department (e.g. marketing, product, engineering). A cross-functional governance committee of data engineers, legal councils, data protection officer and business leaders oversights data management and compliance.

In the context of this organizational model, both full-code and low/no-code solutions will be used.
\begin{itemize}
	\item \textbf{Full-code} technologies are mandatory among the CoE team. They must also be available among the embedded teams to allow for interfacing data manipulation with the CoE team.
	\item \textbf{Low/No-code} technologies are key to the embedded teams in the different department. They provide high data availability for a large volume of non-tech people such as business analysts.
\end{itemize}


\section*{Technologies and tools}

\begin{table}[h]
	\centering
	\begin{threeparttable}
		\caption{Recommended technology tools to support Spotify data governance framework.}
		\label{table:tools}
		\begin{tabularx}{0.99\textwidth}{l >{\arraybackslash}X >{\arraybackslash}X}
			\toprule
			\multicolumn{1}{c}{\textbf{Category}} & \multicolumn{1}{c}{\textbf{Purpose}} & \multicolumn{1}{c}{\textbf{Examples}} \\
			\midrule
			Data cataloging & Central metadata management, data discovery & \href{https://www.alation.com/}{Alation}, \href{https://atlas.apache.org/}{Apache Atlas}, \href{https://atlan.com/}{Atlan}, \href{https://www.collibra.com/}{Collibra} \\
			Data quality & Automate data profiling, validation, alerts & \href{https://www.ataccama.com/}{Ataccama}, \href{https://www.informatica.com/products/data-quality.html}{Informatica Data Quality}, \href{https://www.soda.io/}{Soda}, \href{https://www.talend.com/products/data-integrity-governance/}{Talend} \\
			Compliance monitoring & Manage user consent, automate DSARs & \href{https://bigid.com/}{BigID}, \href{https://www.onetrust.com/}{OneTrust}, \href{https://trustarc.com/}{TrustArc}, \href{https://verasafe.com/}{VeraSafe} \\
			Data security & Protection of sensitive user and payment data & \href{https://www.dataguard.com/}{DataGuard}, \href{https://www.splunk.com/}{Splunk}, \href{https://cpl.thalesgroup.com/encryption/vormetric-data-security-platform}{Vormetric} \\
			Lineage and integration & Track data flow across systems & \href{https://atlas.apache.org/}{Apache Atlas}, \href{https://www.informatica.com/products/data-quality.html}{Informatica} \\
			\bottomrule
		\end{tabularx}
	\end{threeparttable}
\end{table}

It is of course recommended to avoid redundancy by selecting a single tool for a given purpose.


\section*{Pilot implementation}

\subsection*{Overview of the Pilot Implementation}

\textbf{Objective of the Pilot:}\\
To validate the effectiveness of the new Data Governance Framework of Spotify by applying it to the user data domain. The rationale behind this choice is that user data is central to key components of the governance plan: compliance, quality and security. The dataset considered includes personal information (e.g., phone number, banking information) and user interactions (e.g., play history, search behavior, playlist creation). The pilot aims to enhance data quality, ensure GDPR/CCPA compliance, and improve internal accessibility and reliability of engagement insights.

\noindent\textbf{Scope of the Pilot:}\\
The pilot will focus on the user data domain. The rationale behind this choice is that user data is central to key components of the governance plan: compliance, quality and security.


\subsection*{Key Goals}

\begin{itemize}[itemsep=5pt, parsep=0pt]
	\item \textbf{Improve Data Quality} (accuracy, completeness, consistency) in data records
	\item \textbf{Ensure Compliance} with GDPR/CCPA for user data
	\item \textbf{Streamline Access} to clean, governed data for authorized staff
	\item \textbf{Mitigate Risks} associated with personal data misuse or poor data quality
\end{itemize}


\subsection*{Pilot Team and Roles}

\begin{table}[ht]
	\centering
	\begin{threeparttable}
		%\caption{}
		\label{table:roles}
		\begin{tabularx}{0.99\textwidth}{c l >{\arraybackslash}X}
			\toprule
			\multicolumn{1}{c}{\textbf{Team member}} & \multicolumn{1}{c}{\textbf{Role}} & \multicolumn{1}{c}{\textbf{Responsibilities}} \\
			\midrule
			- & Pilot Project Manager & Oversees pilot execution, coordination, and reporting \\
			- & Data Steward & Manages quality and documentation of engagement data \\
			- & Data Protection Officer (DPO) & Conducts privacy audits and ensures GDPR/CCPA compliance \\
			- & IT Engineer / Data Analyst & Implements quality monitoring and access control tools \\
			- & Product Department Lead & Aligns implementation with product strategy and insights \\
			\bottomrule
		\end{tabularx}
	\end{threeparttable}
\end{table}


\subsection*{Timeline and Milestones}

Important steps of pilot implementation are given below. They should be completed in 3-6 months.

\begin{table}[ht]
	\centering
	\begin{threeparttable}
		%\caption{}
		\label{table:milestones}
		\begin{tabularx}{0.99\textwidth}{c >{\arraybackslash}X >{\arraybackslash}X}
			\toprule
			\multicolumn{1}{c}{\textbf{Milestone}} & \multicolumn{1}{c}{\textbf{Target date}} & \multicolumn{1}{c}{\textbf{Responsible}} \\
			\midrule
			Kick-off meeting & asap & Project Manager \\
			Data Assessment and Cleansing & +1 month & Data Steward \\
			GDPR/CCPA Compliance Audit & +1 month & DPO \\
			Technical Setup and Integration & +2 month & IT Engineer \\
			Mid-Project Review & +2 months & Project Manager \\
			Final Review and Closure & +4 months & Project Manager \\
			\bottomrule
		\end{tabularx}
	\end{threeparttable}
\end{table}


\subsection*{Key Deliverables}

\begin{itemize}[itemsep=5pt, parsep=0pt]
	\item \textbf{Data Quality Report} (before/after metrics for missing/duplicate/inconsistent data)
	\item \textbf{Compliance Assessment Report} of data processing practices
	\item \textbf{Technical Integration Plan} for data across product and analytics teams
	\item \textbf{Risk Assessment Report} for data security and exposure, with mitigation strategies
	\item \textbf{Stakeholder Feedback Summary} with insights from product managers and analysts
\end{itemize}


\subsection*{Key Performance Indicators (KPIs)}

\noindent\textbf{Compliance:}
\begin{itemize}[itemsep=5pt, parsep=0pt]
	\item 100\% GDPR/CCPA consent tracking on the dataset
	\item 0 security incidents involving personal data during the pilot
	\item 90\% service-level agreement compliance on data subject requests
\end{itemize}

\noindent\textbf{Data Quality:}
\begin{itemize}[itemsep=5pt, parsep=0pt]
	\item 10\% reduction in missing or inaccurate user interaction logs
	\item < 3\% data quality error rate in governed domain
\end{itemize}

\noindent\textbf{Performance improvement:}
\begin{itemize}[itemsep=5pt, parsep=0pt]
	\item 20\% faster access to engagement insights via data catalog
	\item 80\% data literacy training adoption in pilot teams
\end{itemize}


\subsection*{Risk Management}

\begin{table}[h]
	\centering
	\begin{threeparttable}
		%\caption{}
		\label{table:risks}
		\begin{tabularx}{0.99\textwidth}{>{\arraybackslash}X c c >{\arraybackslash}X}
			\toprule
			\multicolumn{1}{c}{\textbf{Risk}} & \multicolumn{1}{c}{\textbf{Likelihood}} & \multicolumn{1}{c}{\textbf{Impact}} & \multicolumn{1}{c}{\textbf{Mitigation Strategy}} \\
			\midrule
			Non-compliance with regulations & Medium & High & Weekly audits and DPO signoff \\
			Team resistance to new processes & High & Medium & Stakeholder workshops and clear benefit framing \\
			Unresolved data quality issues & Medium & High & Pre-pilot documentation sprint; Regular monitoring and review \\
			Technical integration complexity & Medium & High & IT involvement in early scoping \\
			\bottomrule
		\end{tabularx}
	\end{threeparttable}
\end{table}


\subsection*{Training and Change Management}

\begin{itemize}[itemsep=5pt, parsep=0pt]
	\item \textbf{Training Sessions:} Conduct onboarding for Product and Analytics teams
	\item \textbf{Resources:} Share data governance guides, FAQs, and a support channel
	\item \textbf{Feedback Loop:} Weekly check-ins and a post-pilot survey to gather input from users
\end{itemize}


\subsection*{Evaluation and Lessons Learned}

\begin{itemize}[itemsep=5pt, parsep=0pt]
	\item \textbf{Evaluation Metrics:} Compare pre/post KPI results and stakeholder feedback
	\item \textbf{Lessons Learned:} Identify areas of resistance, gaps in metadata, or unanticipated integration issues
	\item \textbf{Feedback Summary:} Compile suggestions to refine the framework before full rollout
\end{itemize}


\subsection*{Next Steps}

\begin{itemize}[itemsep=5pt, parsep=0pt]
	\item \textbf{Scaling Plan:} Expand framework to cover the other data domains (e.g., Marketing Data and Content Metadata)
	\item \textbf{Adjustments:} Update standards and training materials based on pilot outcomes
	\item \textbf{Full Rollout Proposal:} Present refined strategy and timeline to the Data Governance Committee
\end{itemize}

%\subsection{nice}
%
%Spotify will begin with a pilot implementation in the user data domain. The rationale behind this choice is that user data is central to key components of the governance plan: compliance, quality and security. The steps of this pilot implementation are given below. They should be completed in 3-6 months.
%\begin{itemize}[itemsep=5pt, parsep=0pt]
%	\item[\textbf{1.}] Select Stakeholders: CDO, Data Stewards (Marketing, Product), DPO, Legal.
%	\item[\textbf{2.}] Audit Existing Data: Identify sources, access logs, quality gaps.
%	\item[\textbf{3.}] Apply Governance Rules: Define metadata, access controls, validation checks.
%	\item[\textbf{4.}] Configure Tools: Deploy data catalog and privacy software for pilot scope.
%	\item[\textbf{5.}] Train Teams: Deliver sessions on new policies, roles, and tools.
%	\item[\textbf{6.}] Monitor Metrics: \% of high-quality user records, of DSARs handled on time, user trust/engagement metrics
%	\item[\textbf{7.}] Review \& Iterate: Adjust processes before scaling to other domains.
%\end{itemize}
%
%The pilot implementation will have the following target performance indicators:
%\begin{itemize}[itemsep=5pt, parsep=0pt]
%	\item 90\% service-level agreement compliance on data subject requests
%	\item 80\% data literacy training adoption in pilot teams
%	\item less than 3\% data quality error rate in governed domains
%\end{itemize}


\end{document}